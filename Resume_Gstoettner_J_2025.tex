%%%%%%%%%%%%%%%%%%%%%%%%%%%%%%%%%%%%%%%%%
% Medium Length Professional CV
% LaTeX Template
% Version 3.0 (December 17, 2022)
%
% This template originates from:
% https://www.LaTeXTemplates.com
%
% Author:
% Vel (vel@latextemplates.com)
%
% Original author:
% Trey Hunner (http://www.treyhunner.com/)
%
% License:
% CC BY-NC-SA 4.0 (https://creativecommons.org/licenses/by-nc-sa/4.0/)
%
%%%%%%%%%%%%%%%%%%%%%%%%%%%%%%%%%%%%%%%%%

%----------------------------------------------------------------------------------------
%	PACKAGES AND OTHER DOCUMENT CONFIGURATIONS
%----------------------------------------------------------------------------------------

\documentclass[
	a4paper, % Uncomment for A4 paper size (default is US letter)
	11pt, % Default font size, can use 10pt, 11pt or 12pt
]{resume} % Use the resume class

\usepackage{ebgaramond} % Use the EB Garamond font
\usepackage{hyperref} % Required for hyperlinks

%------------------------------------------------

\name{Josef Gstoettner} % Your name to appear at the top

% You can use the \address command up to 3 times for 3 different addresses or pieces of contact information
% Any new lines (\\) you use in the \address commands will be converted to symbols, so each address will appear as a single line.

% \address{123 Broadway \\ City, State 12345} % Main address

% \address{123 Pleasant Lane \\ City, State 12345} % A secondary address (optional)

\address{(852)~$\cdot$~9322~$\cdot$~5289 \\ jgstoettner@connect.ust.hk} % Contact information

\address{ \url{https://github.com/JosefGst} \\ \url{https://josefgst.github.io/}} % GitHub profile (optional)

%----------------------------------------------------------------------------------------

\begin{document}

\textbf{Builder} and \textbf{Tinkerer} experienced with ROS, especially in development of autonomous mobile robots.
% Adept at collaborating with cross-functional teams to bring projects from concept to completion. 

%----------------------------------------------------------------------------------------
%	TECHNICAL STRENGTHS SECTION
%----------------------------------------------------------------------------------------

\begin{rSection}{Skills}

	\item ROS, C++, Python, Docker, Matlab
	\item Embedded software development on ESP32, Arduino, STM32, NRF52
	\item CAD design for 3D printing, laser cut and tool path generation for CNC machining in Fusion 360

\end{rSection}


%----------------------------------------------------------------------------------------
%	LANGUAGES SECTION
%----------------------------------------------------------------------------------------

\begin{rSection}{Laguanges}
	\item German (native), English (fluent), Chinese Mandarin (intermediate)

\end{rSection}


%----------------------------------------------------------------------------------------
%	EDUCATION SECTION
%----------------------------------------------------------------------------------------

\begin{rSection}{Education}
	
	\textbf{HKUST} \ Master in Mechanical Engineering \hfill \textit{May 2019}

	
\end{rSection}

%----------------------------------------------------------------------------------------
%	WORK EXPERIENCE SECTION
%----------------------------------------------------------------------------------------

\begin{rSection}{Experience}

	\begin{rSubsection}{Full time Parent}{September 2024 - Present}{}{}
		\item Perhaps the hardest job in the world.
	\end{rSubsection}

%------------------------------------------------

	\begin{rSubsection}{LSCM --- Robotic Engineer, ROS}{April 2022 - August 2024}{}{}
		\item Developed motor drivers and for ROS robots in C++ and Python.
		\item Integrated sensors like, Lidar, IMU, depth Cam and RGB cameras for navigation and mapping.
		\item Created simple GUI's for delivery robots.
	\end{rSubsection}

%------------------------------------------------

	\begin{rSubsection}{HKUST --- Research Assistant, Embedded Software}{July 2020 - March 2022}{}{}
		\item Developed a weight scale with RFID scanner for automated storage records in chemical Labs on Arduino MCU.
		\item CAD design for 3D print and laser cut of the prototypes.
		\item Worked on a low power IoT accelerometer with BLE Mesh for predictive maintenance.
	\end{rSubsection}

%------------------------------------------------

	\begin{rSubsection}{KALBAS --- CAD Designer, Product development}{August 2019 - May 2020}{}{}
		\item Designed, 3D-printed and created tool-paths for CNC machining of fish lure prototypes.
	\end{rSubsection}

\end{rSection}


%----------------------------------------------------------------------------------------
%	PROJECTS SECTION
%----------------------------------------------------------------------------------------

\begin{rSection}{Projects}

	\begin{rSubsection}{\href{https://github.com/JosefGst/lingao_ros2}{Lingao ROS 2}; Private}{August 2023}{}{}
		\item Convert the existing ROS 1 codebase to ROS 2 of the Lingao robot.
		\item Add outdoor navigation with GPS.
	\end{rSubsection}

%------------------------------------------------

	\begin{rSubsection}{Red Bird Racing; Autonomous Racing Team; HKUST}{November 2021 - April 2022}{}{}
		\item Cone detection with OpenCV and autonomous race car control-algorithm in ROS.
	\end{rSubsection}

%------------------------------------------------

	\begin{rSubsection}{Robomaster; Software team; HKUST}{October 2021 - April 2022}{}{}
		\item SLAM for autonomous Robot in ROS and embedded software development on STM32.
	\end{rSubsection}

%------------------------------------------------

	\begin{rSubsection}{\href{https://github.com/JosefGst/autorace}{Autonomous RC-car challenge} (first place); HKUST}{December 2020 - March 2021}{}{}
		\item Trained Pytorch model on the Jetson Nano for autonomous-driving, obstacle avoidance and overtaking other cars.
	\end{rSubsection}

\end{rSection}


%----------------------------------------------------------------------------------------
%	EXAMPLE SECTION
%----------------------------------------------------------------------------------------

% \begin{rSection}{Section Name}

% 	Section content\ldots

% \end{rSection}

%----------------------------------------------------------------------------------------

\end{document}
