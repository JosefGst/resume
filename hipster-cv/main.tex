\documentclass[grey]{hipstercv}
% available options are: darkhipster, lighthipster, pastel, allblack, grey, verylight
\usepackage[utf8]{inputenc}
\usepackage[default]{raleway}
\usepackage[margin=1cm, a4paper]{geometry}


%------------------------------------------------------------------ Variablen

\newlength{\rightcolwidth}
\newlength{\leftcolwidth}
\setlength{\leftcolwidth}{0.3\textwidth}
\setlength{\rightcolwidth}{0.65\textwidth}

%------------------------------------------------------------------
\title{Hipster-CV}
\author{\LaTeX{} Ninja}
\date{March 2019}

\pagestyle{empty}
\begin{document}


\thispagestyle{empty}
%-------------------------------------------------------------

\section*{Start}



\header{\bg{headerfontbox}{headerfontboxfont}{Robotics Engineer} }{\bgupper{headerfontbox}{headerfontboxfont}{\bfseries\Huge Josef Gstoettner}}{\bg{headerfontbox}{headerfontboxfont}{\large \faGraduationCap~ Masters in Mechanical Engineering HKUST}}{profile_cropt.png}{headerblue}{3.5cm}{2cm}



%------------------------------------------------

% hier muss die "unsichtbare" Überschrift rein, weil er sonst nicht die Paracols startet... komisch...
\subsection*{}
\vspace{4em}

\setlength{\columnsep}{1.5cm}
\columnratio{0.3}[0.65]
\begin{paracol}{2}
\hbadness5000
%\backgroundcolor{c[1]}[rgb]{1,1,0.8} % cream yellow for column-1 %\backgroundcolor{g}[rgb]{0.8,1,1} % \backgroundcolor{l}[rgb]{0,0,0.7} % dark blue for left margin

\paracolbackgroundoptions

% 0.9,0.9,0.9 -- 0.8,0.8,0.8


\footnotesize
{\setasidefontcolour
\bgupper{cvgreen}{white}{Info} \\
\bg{cvgreen}{white}{Contact} \\

\begin{tabular}{ll}
\faPhone& +852 9322 5289 \\
\faAt& \protect\href{mailto:jgstoettner@connect.ust.hk}{jgstoettner@connect.ust.hk} \\
\faGithub& \protect\url{https://github.com/JosefGst} \\
\faRss& \protect\url{https://josefgst.github.io/blog/} \\
\faLinkedin& \protect\href{https://www.linkedin.com/in/josef-gstoettner-437630172/}{Linkedin} \\

% \faGlobe& nationality: English  \\
% \faBirthdayCake&1690 \\
% \faMapMarker&on a ship \\
\end{tabular}

\bigskip

\bgupper{cvgreen}{white}{Skills} \\

\bg{cvgreen}{white}{Areas of specialization} \\

 - ROS / ROS 2 for mobile robots \\
 - Robot Simulations in Gazebo \& Isaacsim\\
 - Embedded software for ESP32, Arduino, STM32, NRF52 \\
 - CAD design in Fusion360 \& Solidworks\\
 - 3D printing \\
 - CNC machining and laser cutting 

\bigskip

% \vspace{4em}

\bg{cvgreen}{white}{Learning \& Hobbies} \\

- PCB design with KiCAD \\
- Game development in Godot \\ 
- Mandarin and Cantonese

\bigskip
% \vspace{4em}

\bg{cvgreen}{white}{Languages}
\bigskip


\begin{minipage}[t]{\leftcolwidth}
\begin{tabular}{l | ll}
\textbf{English} & {\phantom{x}\footnotesize fluenluent} \\
\textbf{Mandarin} & {\phantom{x}\footnotesize conversational} \\
\textbf{Cantonese} & {\phantom{x}\footnotesize basic} \\
\textbf{German} & {\phantom{x}\footnotesize native} \\
\end{tabular}
\end{minipage}

\bigskip

% \bg{cvgreen}{white}{Interests}\\

% \bubblediagram{{\textbf{a pirate's} \\\textbf{life}},  the sea, pillaging, plundering, ships, stealing, hijacking, \textbf{The Black}\\ \textbf{Pearl}}


\bg{cvgreen}{white}{IT \& programming} \\

\begin{minipage}[t]{0.3\textwidth}
    \begin{tabular}{r @{\hspace{0.5em}}l}
        \bg{skilllabelcolour}{iconcolour}{python} & \barrule{0.5}{0.5em}{cvpurple} \\
        \bg{skilllabelcolour}{iconcolour}{C / C++} & \barrule{0.5}{0.5em}{cvpurple} \\
        \bg{skilllabelcolour}{iconcolour}{C\#} & \barrule{0.4}{0.5em}{cvpurple} \\
        \bg{skilllabelcolour}{iconcolour}{Matlab} & \barrule{0.4}{0.5em}{cvpurple} \\
        \bg{skilllabelcolour}{iconcolour}{Docker} & \barrule{0.4}{0.5em}{cvpurple} \\
        \bg{skilllabelcolour}{iconcolour}{html, css} &  \barrule{0.3}{0.5em}{cvpurple}\\
        \bg{skilllabelcolour}{iconcolour}{javascript} & \barrule{0.3}{0.5em}{cvpurple} \\
        %  \bg{skilllabelcolour}{iconcolour}{\LaTeX} & \barrule{0.55}{0.5em}{cvgreen} \\
    \end{tabular}
    
    
    %\dashrule{}{}
\end{minipage}

\hspace{3cm} \color{labelcolour}{OS:} \hspace{0.5em}\icon{\faWindows}{labelcolour}{\Large} \hspace{0.5em} \icon{\faLinux}{labelcolour}{\Large} 
\bigskip


\bigskip

% \scalebox{0.8}{
% \iconcross{\Huge}{white}{cvred}{\color{black!30}\faBook}{\href{mailto:the.latex.ninja@gmail.com}{\faEnvelopeO}}{\faPhone}{\faCode}
% }

\phantom{turn the page}

\phantom{turn the page}
}
%-----------------------------------------------------------
\switchcolumn

C++ and Python developer with experience in ROS. Makes robots navigate autonomously. Well rounded
mechatronics engineer. Can work on software, mechanics and electronics.

\small
\section*{Short Resumé}

\begin{tabular}{r| p{0.4\textwidth} c}
    \cvevent{2022--2024}{System Engineer}{ROS Robotics}{LSCM\color{cvred}}{Set up SLAM (cartographer, slamtoolbox, rtabmap) and NAV2 on autonomous mobile robots.  \newline Developed motor drivers and autonomous docking in C++ and Python. \newline Experience with wide range of sensors (3D LiDAR, depth cameras, IMU, GPS, Sonar.).}{assets/LSCM.jpg} \\
    \cvevent{2020--2022}{Research Assistant}{Embedded Software}{HKUST \color{cvred}}{Developed a weight scale with RFID scanner for automated storage records in chemical Labs on Arduino MCU. \newline Firmware development on a low power IoT accelerometer with BLE Mesh for predictive maintenance based on Nrf52.}{assets/ust.png} \\
    \cvevent{2019--2020}{Mechanical Engineer}{CAD design}{KALBAS \color{cvred}}{Designed, 3D-printed and created tool-paths for CNC machining of fish lure prototypes.}{assets/empty.png}
\end{tabular}

% \vspace{4em}

\section*{Projects}

\begin{tabular}{r| p{0.4\textwidth} c}
    \cvevent{2023}{Lingao ROS 2}{ROS 2}{Personal side Project}{\faGithub~ \url{https://github.com/JosefGst/lingao_ros2} \newline Build an autonomous mobile robot from scratch for outdoor environment.}{assets/empty.png} \\
    \cvevent{2021--2022}{Red Bird Racing}{Autonomous Racing}{HKUST \color{cvred}}{Cone detection with OpenCV and autonomous race car control-algorithm in ROS}{assets/red_bird.png} \\
    \cvevent{2021--2022}{Robomaster}{Software team}{HKUST \color{cvred}}{SLAM and navigation for autonomous Robot in ROS and embedded software development on STM32.}{assets/robo_master.jpg} \\
    \cvevent{2020--2021}{Autonomous RC-car race (first place \faTrophy)}{Imitation Learning}{HKUST \color{cvred}}{\faGithub~ \protect\url{https://github.com/JosefGst/autorace} \newline Trained Pytorch model on the Jetson Nano for autonomous-driving, obstacle avoidance and overtaking of other cars.}{assets/autorace.png}
\end{tabular}

% \vspace{4em}


% \begin{minipage}[t]{0.4\textwidth}
% \section*{Degrees}
% \begin{tabular}{r p{0.6\textwidth} c}
%     \cvdegree{1710}{Captain}{Certified}{Tortuga Uni \color{headerblue}}{}{disney.png} \\
%     \cvdegree{1715}{Bucaneering}{M.A.}{London \color{headerblue}}{}{medal.jpeg} \\
%     \cvdegree{1720}{Bucaneering}{B.A.}{London \color{headerblue}}{}{medal.jpeg}
% \end{tabular}
% \end{minipage}\hfill
% \begin{minipage}[t]{0.16\textwidth}
% \section*{Hobbies}
% % usage \hobbyicon{<fontawesome icon}{Text}{background color of circle}{size of icon}{space text below icon}
% \hobbyicon{\color{iconcolour}\faFlask}{Rhum}{cvgreen}{\iconsize}{2em} \hfill
% \hobbyicon{\color{iconcolour}\faBook}{The Code}{cvorange}{\iconsize}{2em}

% \hobbyicon{\color{iconcolour}\faComment}{Parler}{cvpurple}{\iconsize}{2em} \hspace{1em}
% \hobbyicon{\color{iconcolour}\faBeer}{Beer}{headerblue}{\iconsize}{2em}
% \end{minipage}

\vspace{1em}

% \begin{minipage}[t]{0.3\textwidth}
% \section*{Certificates \& Grants}
% \begin{tabular}{>{\footnotesize\bfseries}r >{\footnotesize}p{0.55\textwidth}}
%     1708 & Captain's Certificates \\
%     1710 & Travel grant \\
%     1715--1716 & Grant from the Pirate's Company
% \end{tabular}
% \section*{Strenghts}
% \cvtag{honest}
% \cvtag{thieving}
% \cvtag{handsome}
% \section*{References}
% \cvkeyword{Will Turner}{cvgreen}{iconcolour}
% \cvkeyword{Barbossa}{cvgreen}{iconcolour} \\

% \cvkeyword{possibly Mr. Swan}{headerblue}{iconcolour}
% \end{minipage}\hfill
% \begin{minipage}[t]{0.3\textwidth}
% \section*{Publications}
% \begin{tabular}{>{\footnotesize\bfseries}r >{\footnotesize}p{0.7\textwidth}}
%     1729 & \emph{How I almost got killed by Lady Swan}, Tortuga Printing Press. \\
%     1720 & ``Privateering for Beginners'', in: \emph{The Pragmatic Pirate} (1/1720).
% \end{tabular}
% \section*{Talks}
% \begin{tabular}{>{\footnotesize\bfseries}r >{\footnotesize}p{0.6\textwidth}}
%     Nov. 1726 & ``How I lost my ship (\& and how to get it back)'', at: \emph{Annual Pirate's Conference} in Tortuga, Nov. 1726.
% \end{tabular}
% \end{minipage}









\vfill{} % Whitespace before final footer

%----------------------------------------------------------------------------------------
%	FINAL FOOTER
%----------------------------------------------------------------------------------------
\setlength{\parindent}{0pt}
\begin{minipage}[t]{\rightcolwidth}
\begin{center}\fontfamily{\sfdefault}\selectfont \color{black!70}
{\small Josef Gstoettner \icon{\faPhone}{cvpurple}{} +852 9322 5289 \icon{\faAt}{cvpurple}{} \protect\url{jgstoettner@connect.ust.hk} \\ 
\icon{\faGithub}{cvpurple}{} \protect\url{https://github.com/JosefGst} \icon{\faRss}{cvpurple}{} \protect\url{https://josefgst.github.io/blog/} \\ 
\icon{\faLinkedin}{cvpurple}{} \protect\url{https://www.linkedin.com/in/josef-gstoettner-437630172/} 
}
\end{center}
\end{minipage}


\end{paracol}

\end{document}